\documentclass[aps,letterpaper,10pt]{revtex4}

\usepackage{graphicx}
\usepackage{amsmath}


\begin{document}


\begin{center}
Jeffrey Gertler \\ \vspace{14pt}
Advisor: Prof. David Hogg \\ \vspace{14pt}

\end{center}

%%%%%%%%%%%%
\section{Introduction}
\subsection{Objective}
Create a set of data that approximates the times of photon emissions caused by a given gamma $\frac{photons}{time}$ value over a given $\Delta t$. Also allow for data sets where the gamma changes with time.


\section{Current Solution}


\subsection{Implementing a Changing Gamma}
For a gamma that changes over time the main concept of the data generation is the same but broken into smaller sections. To find the ideal number of photons in a given  $\Delta t$ the program simply multiplied the gamma by the  $\Delta t$ which is the same as the integral of that linear function over  $\Delta t$. Similarly, for any function of gamma over time the ideal photon count over that total time will be the integral of that function over the given  $\Delta t$. A good approximation of the integral of a function is the Riemann sum of the function which breaks the integral into a number of small rectangles (this approximation is exact for a linear function).
\vspace{3mm}

With this in mind the problem of a changing gamma can easily be solved by breaking the  $\Delta t$ into many smaller time steps (currently the program uses time steps of 1 second). Over each smaller time step we find an experimental number of photons with the random.poisson() function and generate that number of photon times like before.
\vspace{3mm}

A significant advantage to this method is the ability to quickly adapt the program to non-linear functions. The only line of code that would need to be changed for a different function would be
\begin{verbatim}
idealPNum = int(gamma + (i+.5)*dGamma)
\end{verbatim}
Instead, any function could be plugged into the integer typecast, with the correct command line arguments, and the program would correctly generate photon times.

\vspace{3mm}


\section{Bin Size $\ll1$}
Initially in the implementation of the changing gamma data generation I used Riemann sums with a width or bin size of 1 second. With this bin size, while small, depending on the gamma there could be a large number of photon times within this time step. This is significant because if there is a large number of photons within a given time period then the difference between a poisson random generator (as used with 1 second bins) and a simple random number generator becomes significant. When the bin size is small, to the point that there is only 0 or 1 photon in each bin, the difference between a poisson distribution and a random one is negligible. This shortening of the bin size also has the advantage of eliminating any need to take an integral of the function to increase accuracy.


\section{Probability of Generating a Specific Bin Set}
Now that we have generated the set of data using bins as described above, the task of finding the probability of getting a specific set of data given a specific rate (gamma) becomes significantly easier. Without the small bin sizes it would be necessary to find the probability of a set of data using the Poisson distribution but when there are only 1 or 0 photons in each bin this changes.\\

Assuming there is only 0 or 1 photons in each bin, for any bin the chance of finding 1 photon is the rate of photon hits (gamma) divided by the time interval of the bin. Respectively the chance of finding 0 photons would then be $1\frac{gamma}{\Delta t}$. Now that we have these two possible probabilities for each bin we can combine them by treating them as independent probabilities and multiplying them together. The product of all these probabilities will be the probability of getting this exact set of filled and empty bins.\\

\begin{displaymath}
   P(bin) = \left\{
     \begin{array}{lr}
       \Gamma /\Delta t & : Photon\ \#\ in\ \Delta t = 0\\
       1-\Gamma /\Delta t & : Photon\ \#\ in\ \Delta t = 1
     \end{array}
   \right.
\end{displaymath}

\begin{gather*}
	P_\Gamma = \Pi_{i = 0}^{n} P(bin_i)\\
\end{gather*}

\vspace{3mm} 

\section{Graphical Output}

\begin{figure}[h!]
%\begin{center} \includegraphics[height=100mm]{../code/linearData0.png} \caption{Data output with a slope of 0} \end{center}
\vspace{3mm}
\end{figure}

\begin{figure}[h!]
%\begin{center} \includegraphics[height=100mm]{linearData1.png} \caption{Data output with a slope of 1} \end{center}
\vspace{3mm}
\end{figure}

\begin{figure}[h!]
%\begin{center} \includegraphics[height=100mm]{linearData-1.png} \caption{Data output with a slope of -1} \end{center}
\vspace{3mm}
\end{figure}

\section{Code}

\begin{verbatim}
for i in range(0, int(dt)):
    idealPNum = int(gamma + (i+.5)*dGamma)
    expPNum = np.random.poisson(idealPNum, 1)
    for j in range(0, expPNum):
        random.seed(seed*j*i)
        tData = np.append(tData, random.random() + i)
\end{verbatim}



\end{document} 
